\documentclass[12pt, a4paper]{article}

\usepackage{ucs}
\usepackage[russian]{babel}
\usepackage{cmap}
\usepackage[utf8x]{inputenc}
\usepackage{amsthm}
\usepackage{amsmath}
\usepackage{amssymb}
\usepackage{graphicx}
\usepackage{listings}
\usepackage{float}
\usepackage{clrscode}
\usepackage{tocloft}
\usepackage[usenames]{color}
\usepackage[margin=20mm]{geometry}
\usepackage{sidecap}
\usepackage{url}
\usepackage{hyperref}

\frenchspacing

\thispagestyle{empty}

\begin{document}

\section*{План презентации}

\begin{enumerate}
  \item Содержание презентации
  \item Хемоинформатика, cheminformatics: что за дисциплина
  	 \begin{enumerate}\item Место дисциплины среди других, связь \end{enumerate}
  \item Зарождение хемоинформатики, немного истории, предназначение
  \item Представление данных
	 \begin{enumerate} 
		\item Абстрактный тип данных - помеченый (или молекулярный) граф
		\item Строковое представление - \textsc{SMILES}
		\item Форматы \textsc{MOL, SDF}
	 \end{enumerate}
  \item Некоторые из основных направлений
	 \begin{enumerate}
  		\item Библиотеки соединений, аналог википедии для химии
		\item Виртуальный скрининг, молекулярный докинг
		\item Моделирование экспериментов \emph{in silico}
	 \end{enumerate}
  \item Известные программные продукты
  \item Заключение
  \item Взгляд в будущее, что актуально сейчас для химиков
\end{enumerate}

\section*{Источники}

Труды, успевшие стать фундаментальными в хемоинформатике. Книги.

\begin{enumerate}
  \item F.K. Brown (1998). \emph{«Chapter 35. Chemoinformatics: What is it and How does it Impact Drugs Discovery}
  \item B.A. Bunin, B. Siesel, G.A. Morales, J. Bajorath (2007). \emph{Chemoinformatics: Theory, Practice, \& Products}
  \item A.R. Leach, V.J. Gillet (2007). \emph{An introduction to chemoinformatics}
\end{enumerate}

\noindent Блоги, лекции, презентации, википедия.

\begin{enumerate}
  \item Материалы AACIMP-2008, курс ``Хемоинформатика'' \\ 
	 \url{http://summerschool.ssa.org.ua/}
  \item Дмитрий Павлов,
  	 \emph{Навигация в мире органических соединений}.\\
  	 \url{http://shmat-razum.blogspot.com/2010/07/blog-post.html}
  \item Сергей Кокорин, \emph{Заметки о Cheminfo'S. Strasbourg Summer School on Chemoinformatics.} 
  \item Википедия (русская, английская)
\end{enumerate}

\end{document}

