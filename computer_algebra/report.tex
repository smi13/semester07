\documentclass[a4paper,12pt]{article}

% Encoding support.
\usepackage{ucs}
\usepackage[utf8x]{inputenc}
\usepackage[T2A]{fontenc}
\usepackage[russian]{babel}

\usepackage{graphicx}
\usepackage{listings}

\usepackage{amsmath, amsthm, amssymb}

%\usepackage{indentfirst}

\usepackage{hyperref}

\usepackage[final]{pdfpages}

\frenchspacing
\righthyphenmin=2

\textheight=24cm   
\textwidth=16cm    
\oddsidemargin=0pt 
\topmargin=-1.5cm  
\parindent=24pt    
\parskip=0pt       
\tolerance=2000

\title{Отчет по работе \\ по курсу <<Компьютерная алгебра>> }
\author{Смолов Виктор, Зенцев Фёдор, 4057/2}

\newcommand\Id[1]{\mathfrak{#1}}
\newcommand\BB[1]{\mathbb{#1}}



\begin{document}

\maketitle
\thispagestyle{empty}

\section{Задание}

Пусть задана система $ A $ биномиальных уравнений:

\[
\left\{ 
\begin{array}{l l}
  y_1 - m_1 = 0 \\
  y_2 - m_2 = 0 \\
  \cdots \\
  y_k - m_k = 0
\end{array} \right.
\]
\[m_1, m_2, \cdots, m_k \in M \subset \BB{K}[x_1, x_2, x_3]\]

Требовалось исключить из такой системы $ x_1, x_2, x_3 $, другими словами 
найти $ \Id{\alpha} = (A) \cap \BB{K}[y_1, y_2, \cdots, y_k] $. Затем
найти базис Грёбнера идеала $ \alpha $, который является \emph{торическим}. Последний пункт задания - визуализация
сечения пространства порядков.

\section{Использованный инструментарий}

Использовалась многопользовательская система компьютерной алгебры \begin{tt}SAGE\end{tt} и язык программирования \begin{tt}Python\end{tt}, в частности. Для визуализации был использован пакет \begin{tt}Gfan\end{tt}. 

\newpage

\section{Решения задач и полученные изображения}

\begin{enumerate}
  \item Система: 
    \[ \left\{ 
\begin{array}{l l}
  y_1  -  x_{1}^2 x_2 = 0 \\
  y_2  -  x_{1}^7 x_{2}^5 = 0 \\
  y_3  -  x_{1}^4 x_{2}^3 = 0
\end{array} \right.
\]

Исходный код программы:
\lstset{language=python, caption=1.py,%
label=source-code, basicstyle=\footnotesize,%
numbers=left, numberstyle=\footnotesize, numbersep=5pt, frame=single, breaklines=true, breakatwhitespace=false,%
inputencoding=utf8x}
\lstinputlisting{1.py}

Изображение проекций сечения конусами основного конуса, выводы программы \begin{tt}gfan\_render\end{tt}:

\begin{center}
\includegraphics[scale=0.4]{a.pdf}
\end{center}

\newpage

  \item Система: 
    \[ \left\{ 
\begin{array}{l l}
  y_1  -  x_{1} x_3^2 = 0 \\
  y_2  -  x_{2}^5 = 0 \\
  y_3  -  x_{1}^2 x_{2}^3 = 0 \\
  y_4  -  x_3^3 x_4 = 0 \\
  y_5  -  x_2 x_4^2 = 0 \\
\end{array} \right.
\]

Исходный код программы:
\lstset{language=python, caption=2.py,%
label=source-code, basicstyle=\footnotesize,%
numbers=left, numberstyle=\footnotesize, numbersep=5pt, frame=single, breaklines=true, breakatwhitespace=false,%
inputencoding=utf8x}
\lstinputlisting{2.py}

\begin{center}
\includegraphics[scale=0.06]{mix4.pdf}
\begin{tt}--shiftVariables 0, 1\end{tt}

\includegraphics[scale=0.06]{mix5.pdf}
\begin{tt}--shiftVariables 2, 3\end{tt}

\includegraphics[scale=0.06]{2_4.pdf} \\ 
\begin{tt}--shiftVariables 4\end{tt}
\end{center}

\newpage

\item Система

\[ \left\{ 
\begin{array}{l l}
  y_1 - x_1^3 x_3^5 = 0 \\
  y_2 - x_1^2 x_3 x_2^4 = 0 \\
  y_3 - x_2^2 x_3^3 = 0 \\
  y_4 - x_2^3 x_1 = 0 \\
  y_5 - x_3 x_2 x_1 x_5 = 0 \\
  y_6 - x_4 x_5^2 = 0 \\
  y_7 - x_5 x_2^2 x_1^3 = 0 \\
\end{array} \right. \]

Исходный код программы:
\lstset{language=python, caption=2.py,%
label=source-code, basicstyle=\footnotesize,%
numbers=left, numberstyle=\footnotesize, numbersep=5pt, frame=single, breaklines=true, breakatwhitespace=false,%
inputencoding=utf8x}
\lstinputlisting{2.py}

Изображение проекций сечения конусами основного конуса, вывод программы \begin{tt}gfan\_render\end{tt}:

\begin{center}
\includegraphics[scale=0.06]{mix1.pdf}
\begin{tt}--shiftVariables 0, 1\end{tt}
\end{center}
\newpage

\begin{center}
\includegraphics[scale=0.06]{mix2.pdf}
\begin{tt}--shiftVariables 2, 3\end{tt}

\includegraphics[scale=0.06]{mix3.pdf}
\begin{tt}--shiftVariables 4, 5\end{tt}

\includegraphics[scale=0.06]{3_6.pdf} \\ 
\begin{tt}--shiftVariables 6\end{tt}

\end{center}


\end{enumerate}

\newpage

\section{Результаты}

\begin{enumerate}
  \item 
    \lstset{language=bash, caption=res1.txt,%
label=source-code, basicstyle=\footnotesize,%
numbers=left, numberstyle=\footnotesize, numbersep=5pt, frame=single, breaklines=true, breakatwhitespace=false,%
inputencoding=utf8x}
\lstinputlisting{res1.txt}

\item
   \lstset{language=bash, caption=res2.txt,%
label=source-code, basicstyle=\footnotesize,%
numbers=left, numberstyle=\footnotesize, numbersep=5pt, frame=single, breaklines=true, breakatwhitespace=false,%
inputencoding=utf8x}
\lstinputlisting{res2.txt}

\item
   \lstset{language=bash, caption=res3.txt,%
label=source-code, basicstyle=\footnotesize,%
numbers=left, numberstyle=\footnotesize, numbersep=5pt, frame=single, breaklines=true, breakatwhitespace=false,%
inputencoding=utf8x}
\lstinputlisting{res3.txt}

\end{enumerate}


\end{document}

