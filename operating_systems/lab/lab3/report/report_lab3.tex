\documentclass[a4paper,12pt]{article}

% Encoding support.
\usepackage{ucs}
\usepackage[utf8x]{inputenc}
\usepackage[T2A]{fontenc}
\usepackage[russian]{babel}

\usepackage{graphicx}
\usepackage{listings}

\usepackage{amsmath, amsthm, amssymb}

%\usepackage{indentfirst}

\usepackage{hyperref}

\usepackage[final]{pdfpages}

\frenchspacing
\righthyphenmin=2

\textheight=24cm   
\textwidth=16cm    
\oddsidemargin=0pt 
\topmargin=-1.5cm  
\parindent=24pt    
\parskip=0pt       
\tolerance=2000

\title{Отчет по работе \\ по курсу <<Компьютерная алгебра>> }
\author{Смолов Виктор, Зенцев Фёдор, 4057/2}

\newcommand\Id[1]{\mathfrak{#1}}
\newcommand\BB[1]{\mathbb{#1}}



\begin{document}

\begin{titlepage}

\begin{center}

\large Санкт-Петербургский Государственный Политехнический Университет \\
Кафедра прикладной математики \\ [8.0cm]
\textbf{\textsc{ОТЧЕТ ПО ПРОИЗВОДСТВЕННОЙ ПРАКТИКЕ}}\\[3.0cm]

\begin{minipage}{0.4\textwidth}
\begin{flushleft} \large
  Выполнил студент гр. 3057/2: \\ [1.0cm]
  Руководитель:
\end{flushleft}
\end{minipage}
\begin{minipage}{0.4\textwidth}
\begin{flushright} \large
Зенцев Ф.К. \\ [1.0cm]
Павлов Д.А.
\end{flushright}
\end{minipage}

\vfill

\large Санкт-Петербург 2010



\end{center}
\end{titlepage}


\pagebreak

\textbf{Формулировка задания (3.4):} Напишите программу-демон, получающую в качестве аргумента число N и через N минут после
запуска отправляющую пользователю, от чьего имени она запущена, письмо с произвольным текстом (запустите программу mail). \\

\textbf{Реализация:} На языке \textsc{C}. Для написания программы-демона требуется создать собственный процесс и удалить процесс-родителя, тогда созданный процесс подхватится процессом \textsc{init} - такие процессы и называют демонами. Также для безопасности стоит закрыть
стандартные потоки ввода/вывода. Все равно программа-демон не взаимодействует с терминалом, а оставшиеся открытыми они могут представлять потенциальную опасность. Далее по заданию выжидается время с помощью функции \textsc{sleep}. Сообщение отправляется с помощью вызова функции \textsc{system} - вызывается программа \textsc{mail}. Имя пользователя, запустившего программу, можно узнать с помощью функций \textsc{getuid, getpwuid}.


\section*{Исходный код программы}
\lstset{language=c, caption=Simple daemon program example source, %
label=source-code, basicstyle=\footnotesize,%
numbers=left, numberstyle=\footnotesize, numbersep=5pt, frame=single, breaklines=true, breakatwhitespace=false,%
inputencoding=utf8x}
\lstinputlisting{../src/main.c}

\end{document}

