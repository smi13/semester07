\documentclass{article}%[12pt, a4paper]{article}

\usepackage{ucs}
\usepackage[russian]{babel}
\usepackage{cmap}
\usepackage[utf8x]{inputenc}
\usepackage{amsthm}
\usepackage{amsmath}
\usepackage{amssymb}
\usepackage{graphicx}
\usepackage{float}
\usepackage{clrscode}
\usepackage{tocloft}
\usepackage[usenames]{color}
\usepackage[margin=20mm]{geometry}
\usepackage{sidecap}
\usepackage{url}
\usepackage{hyperref}

\frenchspacing



\begin{document}

\begin{titlepage}

\begin{center}

\large Санкт-Петербургский Государственный Политехнический Университет \\
Кафедра <<Прикладная математика>> \\ [8.0cm]
\textbf{\textsc{ОТЧЕТ ПО ЛАБОРАТОРНОЙ РАБОТЕ №4}}\\[1.0cm]
По курсу ``Операционные системы''\\[3.0cm]

\begin{minipage}{0.4\textwidth}
\begin{flushleft} \large
  Выполнил студент гр. 4057/2: \\ [1.0cm]
  Преподаватель:
\end{flushleft}
\end{minipage}
\begin{minipage}{0.4\textwidth}
\begin{flushright} \large
Зенцев Ф.К. \\ [1.0cm]
Тимофеев Д.А.
\end{flushright}
\end{minipage}

\vfill

\large Санкт-Петербург 2010



\end{center}
\end{titlepage}


\pagebreak

\textbf{Формулировка задания (2.6):}  Напишите простой архиватор (программу, объединяющую несколько файлов в
один), сохраняющий атрибуты файла (владельца, группу, права доступа, дату последнего изменения) и восстанавливающий их при распаковке архива. \\

\textbf{Реализация:} На языке \textsc{C}. Для работы с файловой системой (в образовательных целях) я пользовался системными вызовами: 
\textsc{open, write, read, close}. Упомяну, что вообще эти функции работают медленнее, чем их аналоги из стандартной библиотеки. Для реализации
сохранения различных атрибутов файла была использована функция \textsc{fstat}, а для восстановления - функции \textsc{fchown, fchmod, utime}.

\section*{Исходный код программы}
\lstset{language=c, caption=fz archiver,%
label=source-code, basicstyle=\footnotesize,%
numbers=left, numberstyle=\footnotesize, numbersep=5pt, frame=single, breaklines=true, breakatwhitespace=false,%
inputencoding=utf8x}
\lstinputlisting{../src/main.c}

\end{document}

