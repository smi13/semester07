\documentclass{article}%[12pt, a4paper]{article}

\usepackage{ucs}
\usepackage[russian]{babel}
\usepackage{cmap}
\usepackage[utf8x]{inputenc}
\usepackage{amsthm}
\usepackage{amsmath}
\usepackage{amssymb}
\usepackage{graphicx}
\usepackage{float}
\usepackage{clrscode}
\usepackage{tocloft}
\usepackage[usenames]{color}
\usepackage[margin=20mm]{geometry}
\usepackage{sidecap}
\usepackage{url}
\usepackage{hyperref}

\frenchspacing



\begin{document}

\begin{titlepage}

\begin{center}

\large Санкт-Петербургский Государственный Политехнический Университет \\
Кафедра <<Прикладная математика>> \\ [8.0cm]
\textbf{\textsc{ОТЧЕТ ПО ЛАБОРАТОРНОЙ РАБОТЕ №4}}\\[1.0cm]
По курсу ``Операционные системы''\\[3.0cm]

\begin{minipage}{0.4\textwidth}
\begin{flushleft} \large
  Выполнил студент гр. 4057/2: \\ [1.0cm]
  Преподаватель:
\end{flushleft}
\end{minipage}
\begin{minipage}{0.4\textwidth}
\begin{flushright} \large
Зенцев Ф.К. \\ [1.0cm]
Тимофеев Д.А.
\end{flushright}
\end{minipage}

\vfill

\large Санкт-Петербург 2010



\end{center}
\end{titlepage}


\pagebreak

\textbf{Формулировка задания (4.3):} Напишите программу, моделирующую решение задачи «спящего брадобрея». Задачу необходимо решить
с помощью потоков \textsc{POSIX}. \\

\textbf{Формулировка задачи:} Парикмахерская состоит из \textsc{n} стульев и стула, где брадобрей работает с посетителем. Если
посетителей нет, то брадобрей ложится спать. Если заходит посетитель, но все стулья заняты, он уходит. Если брадобрей занят, но
свободный стул есть, тогда посетитель занимает свободный стул. Если же брадобрей спит, то посетитель будит его и садится в кресло,
где его будут стричь. \\ 

\textbf{Реализация:} На языке \textsc{C} c использованием \textsc{pthreads}. Реализация почти полностью соответствует описанию
решения задачи из книги \textit{The Little Book of Semaphores, Allen B. Downey}.

\section*{Исходный код программы}
\lstset{language=c, caption=Barbershop problem simulation program source, %
label=source-code, basicstyle=\footnotesize,%
numbers=left, numberstyle=\footnotesize, numbersep=5pt, frame=single, breaklines=true, breakatwhitespace=false,%
inputencoding=utf8x}
\lstinputlisting{../src/main.c}

\end{document}

